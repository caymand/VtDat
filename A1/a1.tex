% Created 2022-05-06 Fri 20:23
% Intended LaTeX compiler: pdflatex
\documentclass[11pt]{article}
\usepackage[utf8]{inputenc}
\usepackage[T1]{fontenc}
\usepackage{graphicx}
\usepackage{longtable}
\usepackage{wrapfig}
\usepackage{rotating}
\usepackage[normalem]{ulem}
\usepackage{amsmath}
\usepackage{amssymb}
\usepackage{capt-of}
\usepackage{hyperref}
\author{Kristoffer Kortbaek (jnq578, hold 5)}
\date{\today}
\title{A1}
\hypersetup{
 pdfauthor={Kristoffer Kortbaek (jnq578, hold 5)},
 pdftitle={A1},
 pdfkeywords={},
 pdfsubject={},
 pdfcreator={Emacs 29.0.50 (Org mode 9.6)}, 
 pdflang={English}}
\begin{document}

\maketitle
\tableofcontents


\section{1}
\label{sec:org928e6f6}
Frøkjær, lektor i datalogi på Københavns Universitet, påpeger i sit debatindlæg
``Finansministeriet''\cite{fk} en række
årsager til en række IT skandaler hos SKAT. Hans første udpegning omhandler, at
skandalerne starter efter et krag om at sænke udgifter i skatteadministrationen.
En af måderne dette kan opfyldes, er ved at indføre en personalereduktion på 25\%.
Frøkjær mener, at dette kun kunne lade sig gøre hvis en række dårligt vedligeholdte
IT systemer blev ``saneret''. Herefter kommer Frøkjærs første egentlige pointe. Det
lykkedes kun en andel af IT-saneringsprojekterne at blive gennemført, men planen
om at lave personalereduktion fortsætter på trods af dette.
Frøkjær sætter spørgsmål ved hvordan dette kunne lade sig gøre, og han nævner så
ledes. Frøjkær mener selv, at
der burde blive stillet spørgsmålstegn til hvilken rolle finansministeriet har
spillet skandalerne hos SKAT.

Dernæst kommer Frøkjærs anden pointe. Det var Finansministeriet der startede
personalereduktionen på trods af manglende IT infrastruktur. Han påpeger desuden
at det var Finansministeriet der pressede på med inddrivelsessystemet EFI, og han
påpeget desuden det skulle have en urealistisk grad af automatisering. Endvidere nævnes at dette ikke er første
gange at Finansministeriet angiveligt har presset på med implementering af IT
systemer der har skulle implementere urealistisk grad af automatisering.

\section{2}
\label{sec:orgb8d873b}
Artiklen ``Why Did Your Project Fail''\cite{cerpaverner} generaliserer en række
punkter til ofte at være medskyldige i fejlende IT projekter, og kalder disse for
fejl faktorer. Dette var gjort ved
at udsende et spørgeskema med 88 spørgsmål omkring forskellige aspekter af
softwareudvikling. Svarene fra spørgskemaet blev besvaret af 304 forskellige
projekter, hvoraf
70 af dem blev kategoriseret som værende fejlagtige IT projekter. Herefter blev
der udarbejdet en rangering af de hyppigst forekommende grunde til at IT
projekterne fejler. Disse inkluderer, leveringsdato påvirker udviklingsprocessen,
manglende risikostyring gennem hele projektet, dårlig projekt estimering og
flere. Desuden lægger artiklen vægt på forskellen mellem grundende til fejlende
IT projekter afhængigt af om projektet blev udviklet in-house eller outsourced,
da der angiveligt er forskel i disse. Mangel på kompensation for overarbejde
giver blandt andet anledning til fejlende IT projekter for in-house udvikling,
hvorad dette ikke er et problem når udviklingen var outsourced.

Artiklen nævner derudover at kategoriseringerne af elementer der er fælles for
fejlende IT projekter danner et netværk af elementer der bidrager til at IT
projektet kan fejle. Eksempelvis medvirker at projektet var underestimeret at
flere medarbejdere var tilføjet til projektet sent, hvilket i sig selv er en
fejl faktor som nævnes i artiklen.


\section{3}
\label{sec:orgb4e622a}

Frøkjær påpeger tidligt i sit debatindlæg, hvorfor SKAT fortsatte
personalereduktionen efter det allerede var klart at IT-saneringsprojekterne ikke
ville blive færdiggjort. Herved påpeger han selvsamme fejl faktor som Cerpa og
Verner påpeger i ``Why Did Your Project Fail?''\cite{cerpaverner}, når de
beskriver risici der ikke bliver genovervejet, kontrolleret eller styret gennem
projektet. I følge Cerpa og Verner kan denne fejl faktor hænge sammen med,
at IT-projektet ikke har haft nok medarbejdere, og et ubehageligt arbejdsmiljø. Deres
model siger dog ikke noget om hvordan disse tre fejl faktorer påvirker hinanden,
men blot at der er en sammenhæng mellem dem. Det vil altså sige, at vi ved ikke
om IT-projektet hos SKAT manglede medarbejdere, som derfor førte til dårlig risikostyring, eller
den dårlige risikostyring førte til manglende medarbejdere. Dette minder på sin
vis også om måden at se teknologi som systemer. I stedet for at se teknologi som
systemer som beskrevet af Hughes\cite{hughes}, så ser vi på typiske elementer tilfælles for fejlende IT projekter som
et system. Heri består noderne af netværket af fejl faktorer, og ændringen i en
af disse fejl faktorer kan påvirke resten af netværket.

Frøkjær kritiserer også Finansministeriets krav omkring mængden af automatisering
der skulle være fundet sted i gældsindrivelsessystemet EFI. Dette ligger også op
til fejl faktoren omkring urealistisk forventning\cite{cerpaverner}. I deres
netværk af fejl faktorer, er denne særligt koblet til mange andre faktorer, som netop
antyder at dette problem som Frøkjær tager op også førte til andre fejl faktorer
i EFI projektet. Cerpa og Verner påpeger blandt andet at urealistisk forventning
kan hænge samme med manglende medarbejdere, som i sej selv fører til at ekstra
medarbejdere bliver tilføjet til projektet sent. Dette hænger også sammen med
Frøkjærs påpegning at  Finansministeriet satte ekstra
bevillinger af til EFI projektet.

\section{4}
\label{sec:org214e44a}
Hvis man skal se på hvilke aktører, der generelt kunne have negativ påvirkning på
IT projekter, så peger Frøjkær hvertfald på eksterne interessenter i hans kritik
af Finansministeriet rolle i skandalen hos skat. Man kan måske komme dette endnu
nærmere, hvis man sætter Frøjkærs pointer overfor begrevet ``Computational
Thinking''\cite{wing} og de fejler faktorer\cite{cerpaverner} Cerpa \& Verner
nævner. Wing nævner nemlig, computational thinking metoden mest er kendt inden for
datalogi og softwareudvikling\cite{wing}, men at denne burde bruges i andre
felter, da den ifølge Wing handler om ``prevention, protection, and recovery from
worst-case scenarios''\cite{wing}. Samtidigt snakker Cerpa \& Verner om fejl
faktoren om manglende revurdering af risici under IT projektet\cite{cerpaverner}.
Hermed kan man slutte, at eksterne interessenter eller interne medarbejdere på et
IT projekt, der ikke netop konstant under IT projektets levetid genovervejer
risici har en negativ påvirking på offentlige og private IT-projekters
gennemførsel.

Hvori Frøjkær beskylder Finansministeriet for fejlene gennemførsel af IT
projekter i SKAT, så kunne man omvendt også tænker sig, at det i stedet handlede om
nogle af de interne arbejdsgange indenfor et IT projekt der får det til at fejle.
Dette kunne eksempelvis være mangel på kompetencer eller disciplin inden for
datalogi. Artiklen ``Who are We''\cite{denning} diskuterer netop datalogi som
profession, og nogle af de kriterier der burde være gældende for profession. Han
nævner blandt andet der skal være grundlæggende metoder for at designe og
implementerer sikre og holdbare software systemer\cite{denning}. En udvidelse af
Denning kunne hermed naturligt være, at der som del af datalogi professionen også
skulle være krav omkring hvordan man opnår gennemførsel af IT-projekter. Sat
sammen med ``Computational Thinking''\cite{wing} der netop handler om kontinuerlig
vurdering, så passer dette naturligt til datalogiens metoder.

\begin{thebibliography}{9}
\bibitem{cerpaverner}
Cerpa, Narciso og June M. Verner (dec. 2009). „Why did your project fail?“ Communications of the ACM, bd. 52, nr. 12, s. 130–134. doi: 10.1145/1610252.1610286.

\bibitem{fk}
Frøkjær, Erik (9. jun. 2017). „Finansministeriet“. Weekendavisen, nr. 23.

\bibitem{wing}
 Jeannette M. Wing. ªComputational Thinking“. Communications of the ACM, bd. 49, nr. 3
(mar. 2006), s. 33—35. DOI: 10.1145/1118178.1118215

\bibitem{denning}
Peter J. Denning. ªThe Profession of IT: Who Are We?“ Communications of the ACM, bd. 44, nr. 2 (feb. 2001), s. 15—19. DOI: 10.1145/359205.359239

\bibitem{hughes}
Hughes, Thomas P. (2012). „The Evolution of Large Technological Systems“. I: The Social
Construction of Technological Systems. New Directions in the Sociology and History of
Technology. Red. af Wiebe E. Bijker, Thomas P. Hughes og Trevor Pinch. The MIT
Press, s. 45–76.

\end{thebibliography}
\end{document}
