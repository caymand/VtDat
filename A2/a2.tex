% Created 2022-05-26 Thu 20:05
% Intended LaTeX compiler: pdflatex
\documentclass[11pt]{article}
\usepackage[utf8]{inputenc}
\usepackage[T1]{fontenc}
\usepackage{graphicx}
\usepackage{longtable}
\usepackage{wrapfig}
\usepackage{rotating}
\usepackage[normalem]{ulem}
\usepackage{amsmath}
\usepackage{amssymb}
\usepackage{capt-of}
\usepackage{hyperref}
\usepackage{minted}
\author{Kristoffer Kortbaek}
\date{\today}
\title{A2}
\hypersetup{
 pdfauthor={Kristoffer Kortbaek},
 pdftitle={A2},
 pdfkeywords={},
 pdfsubject={},
 pdfcreator={Emacs 27.2 (Org mode 9.6)}, 
 pdflang={English}}
\begin{document}

\maketitle

\section{Kuhns paradigmer}
\label{sec:org14775c4}
Indenfor Kuhns videnskabsteori opstår begrebet paradigme. For Kunn er dette et fælles verdenssyn der
er delt blandt en gruppe videnskabsfolk, og som helt konkret, bliver instantieret i en disciplinær
matrix\cite{kap2}. Denne desciplinære matrix kan have for skellige dimensioner eller elementer, som
er med til at afgrænse paradigmet.
Herunder er eksempelvis en hvis metafysisk dimension\cite{kap2}. Dette vil være ting, som hvertfald
set udefra paradigmet, ligner metaforer og analogier som paradigmet er bundet i. Set ud fra Hilberts
program, så kan et metafysisk element være ideen om at atomare operationer i den endelige aritmetik
er sikre\cite{kap3}. Altså, at man kan opnå klare deduktive slutninger ud fra atomare operationer i
matematikken.
Derudover kan den disciplinære matrix være tildelt et særligt sæt af værdier, som binder en større
gruppe videnskabsfolk sammen\cite{kap2}. Netop denne dimension er med til at fremhæve
inkommensurabilitet begrebet, siden man ikke, på en rationel måde, kan stå uden for to paradigmer,
og sammenligne dem med hinanden\cite{kap3}. De vil nemlig have forskellige værdier, som ikke
nødvendigvis kan sammenlignes. Sådan en værdi kunne være tilgangen til vigtigheden af
forklaringsgraden af en teori inden for et paradigme. To paradigmer kan således have helt
forskellige værdisæt, i forhold til hvor vigtigt det er at en teori forklarer noget.

Derudover kan den disciplinære matrix også indeholder en rækker eksemplarer, som karakterisere
paradigmet\cite{kap2}. Dette kunne være klassiske lærebøger og problemstillinger, som der er enighed
om. Indenfor datalogi kunne dette være ``The dragon book'' inden for compiler design eller ``K\&R''
indenfor operativsystemer og C eller ``CLRS'' indenfor algoritmik.

\section{Datalogiens paradigme}
\label{sec:org7557f08}
Et paradigme bliver instantieret i dets disciplinære matrice, og herved kan Turing-maskinen
instantieres i en disciplinær matrix. Før den første Turing-maskine blev bygget, lavede Turing sin
model for hvad det vil sige at beregne et tal\cite{kap3}. Dette må siges at være det metafysisk element, siden
datalogier alle tror på denne tanke omkring Turing-maskinen - den kan udregne alt der er
beregneligt.
På det eksemplariske plan, ligger det klassiske datalogiske problem - halting problemet. Turing var
netop interesseret i om man kunne finde ud af om en given Turing-maskine var cirkelfri\cite{kap3}.
Særligt dette problem på den eksemplariske dimension af den disciplinære dimension, er der særdeles
vigtighed om indenfor datalogi. Der udvikles netop programmeringssprog, som ikke er turing
komplette, eller ikke kan simullerer en Turing-maskine, for at man dermed undfår at de er cirkelfri
og halting-problemet bliver trivielt.

\section{Kvantecomputre}
\label{sec:org2499a42}

\section{Revolutioner i datalogi}
\label{sec:org87ee048}

\begin{thebibliography}{9}
\bibitem{kap2}
Sørensen, Søren Kragh. (2022). „Datalogi og teknologi“. Kapitel 2
\bibitem{kap3}
Sørensen, Søren Kragh. (2022). „Fundamentale modeller og datalogiens teoretiske paradigme“. Kapitel 3

\end{thebibliography}
\end{document}
